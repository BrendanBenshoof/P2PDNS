%\documentclass[11pt,aspectratio=169]{beamer}
\documentclass[11pt]{beamer}
\usetheme{Darmstadt}
%\usecolortheme{dolphin}

%\setbeameroption{show notes}
%\setbeameroption{show only notes}
%\usepackage{pgfpages}
%\pgfpagesuselayout{4 on 1}[a4paper]

\usepackage[utf8]{inputenc}
\usepackage[english]{babel}
\usepackage{amsmath}
\usepackage{amsfonts}
\usepackage{amssymb}
\usepackage{graphicx}
\usepackage{subcaption}
\usepackage{algorithmic}
\captionsetup{compatibility=false}

\title[D$^3$NS]{Distributed Decentralized Domain Name Service}
\author{
Brendan Benshoof \qquad Andrew Rosen \\ \qquad Anu G. Bourgeois \qquad Robert W. Harrison }



\logo{\includegraphics[height=1cm]{figs/logo}}
\institute{Georgia State University}
\date{May 27th, 2016}
%\subject{}
%\setbeamercovered{transparent}
%\setbeamertemplate{navigation symbols}{}



%\AtBeginSection[]
%{
%  \begin{frame}
%    \frametitle{Table of Contents}
%    \tiny{\tableofcontents[currentsection]}
%  \end{frame}
%}

\begin{document}
	
\maketitle

\section{Introduction}
\subsection{Motivation}

\begin{frame}{Current DNS}
	\begin{itemize}
		\item ICANN is the final arbiter on who owns what domain
		\item ICANN maintains and organizes the TLD authoritative name servers
		\item Third party verifiers act to authenticate DNS records 
	\end{itemize}
	
	\begin{figure}
		\centering
		\includegraphics[width=0.6\linewidth]{figs/dns}
		%\caption{DNS operation. Source: ICANN}
		\label{fig:dns}
	\end{figure}
	\begin{center}
		\footnotesize{Source: ICANN}
		
	\end{center}
	
\end{frame}






\begin{frame}{Motivation}
	\begin{itemize}
		\item  Recent events have demonstrated that centralized authorities are not as secure a previously hoped
		\begin{itemize}
			\item There is little cryptographic protection against the subpoena
			\item Poorly constructed laws targeting DNS
			\item SOPA and PIPA would have resulted in DNS blocking and compromised security
		\end{itemize}	
		\item  A distributed approach for authentication is much less vulnerable	
	\end{itemize}
	
	\begin{center}
\includegraphics[width=0.4\linewidth]{figs/seized}
\end{center}

\end{frame}

\note{Two interchangble parts 
	the name ownership verification  (name coin variation)
	Replication system VHash or UrDHT (Record progigation in DHT can only happen with authenication).
	Current system caches whatevers
	Robustess -- Regenerating k replicas 
	
	Two philsophical departure.
	Security is baked in at the replication layer.
	Decentralization 
	
	iF we could rewrite DNS from start.
	
	
	Code we wrote falls back (temporary).
	Prototype worked for .TLD
	8.8.8.8
	
	Ideal use case
	If I want to securely run an instance of this, I run a DHT client that provides a local dns server
	Check authentication.
	
	Incentive problem is moot.  
	People will run it if you write it to do so, because people are lazy.
	IPFS proof of providing/proof of retrievablily  - filecoin.
	
}


\begin{frame}{Related Work}
	Cox \textit{et al.}\footnote{Cox \textit{et al.}, ``Serving DNS using a Peer-to-Peer Lookup Service'' in \textit{Peer-to-Peer Systems}, pp. 155-165, Springer, 2002 } developed DDNS:
	\begin{itemize}
		\item Motivated by problem of expertise
		\item Fault tolerant, load-balancing, and scalable
		\item Easier to administer
		\item Found higher latencies in a P2P-based DNS
		\item Incentive problem --  why store records for others?
	\end{itemize}
\end{frame}

\subsection{Big Picture}
\begin{frame}{What is D$^3$NS}
	Distributed Decentralized Domain Name Service	
	\begin{itemize}
		\item Our Goals:
		\begin{itemize}
			
			\item Decentralized authentication
			
			\item Low latency
			\item Incentive to participate
			
			\item Backwards compatible
			\item Transparent to the user
		\end{itemize}
		%	\item Requires distributed signing service for:
		%	\begin{itemize}
		%		\item Authentication
		%		\item Thwarting man-in-the-middle attacks
		%	\end{itemize}
	\end{itemize}
	
\end{frame}


\note[itemize]{
	\item Authenticate records in a distributed system, verisign holds all of the .com
	\item Backwards compatable with the current system
}




\begin{frame}{P2P-Based DNS}
	
	\begin{itemize}
		\item The shared block chain is the final arbiter of who owns what
		\item The DHT organizes and maintains the authoritative TLD servers
		\item The block chain acts to authenticate DNS records
	\end{itemize}
\end{frame}



\section{Implementation}

\subsection{How We Built It}

\begin{frame}{Components}
	
	
	\begin{itemize}
		\item Domain Name Blockchain 
		\item Distributed Hash Table (UrDHT)
		\item DNS server frontend (PowerDNS)
	\end{itemize}
	
	\begin{figure}
		\centering
		\includegraphics[width=\linewidth]{arch}
		\label{fig:arch}
	\end{figure}
	
\end{frame}


\note[itemize]{
	\item Use a variant of NameCoin's blockchain to secure shared list of keys and domains.	
	\item Use the DHT to load balance and distribute responsibility for hosting DNS and keys.  Can be any DHT, we're using UrDHT again.
}




\subsection{DHTs}
%Cuttable
\begin{frame}{Distributed Hash Tables}

	\begin{itemize}
		\item  Means of organizing communication and responsibility in a P2P network
		\item  Each peer is responsible for a verifiable span of hash values
		\item  Facilitates one-to-one communication and one-to-many communication
	\end{itemize}

	\begin{center}
\includegraphics[width=0.4\linewidth]{figs/CR_overlay}
\end{center}


\end{frame}
%
%\begin{frame}{UrDHT}
%	\begin{itemize}
%		\item Abstract DHT.
%		\item Handles:
%		\begin{itemize}
%			\item Arbitrary topology.
%			\item Plugin Services
%		\end{itemize}
%		\item Users can create or run clients to interact with the nodes.
%		\item Subject of other research.
%	\end{itemize}
%	
%\end{frame}

\begin{frame}{UrDHT}
% Cuttable
	\begin{itemize}
		\item Uses Voronoi regions on an $n$-dimensional torus to assign responsibility
		\item Can define how to compute the regions to emulate almost any DHT topology
		\item Node responsibility:
		\begin{itemize}
			\item Node is responsible for its space, defined by its neighbors
			\item If a node leaves/fails, each neighbors assumes that it is responsible until corrected by maintenance 
		\end{itemize}
	\end{itemize}
	
\begin{center}
\includegraphics[width=0.5\linewidth]{figs/new_voronoi}
\end{center}


\end{frame}

\begin{frame}{Fault Tolerance}
	\begin{itemize}
		\item Churn creates a period where I/O can fail
		\item With UrDHT:
		\begin{itemize}
			\item Reads of backed up data are successful
			\item Writes to the region are successful
			\item Reads of \textbf{new} data are vulnerable until it is backed up
			\item This means a much smaller window of vulnerability.  Writes never fail.%\footnote{They may occur out of order}
		\end{itemize}
	\end{itemize}

\end{frame}

\note{Reads of NEW data are vulnerable until they are backed up in the network.}

\begin{frame}{Cool Things UrDHT Can Do}
	\begin{itemize}
		\item Embed problem spaces into DHT topology
		\item Minimal latency based routing
		\item Basically turns routing into best-search first
		
	\end{itemize}
\end{frame}



\subsection{Blockchain}
%\begin{frame}{DNS Blockchain}
%% Cuttable
%	\begin{itemize}
%		\item Based on the blockchain verification of Bitcoin
%		\item Allows for a shared, immutable and secure public records
%		\item One block can include the validation of a new server's public key
%		\item One block can include a DNS record or change
%		\item Blocks require a proof of work to authenticate, causing records to be produced at a semi-fixed rate.
%	\end{itemize}
%\end{frame}

\begin{frame}{DNS Blockchain}
	\begin{itemize}
		\item Using a technique similar to Bitcoin, we can assign domain names as reward for mining new blocks and transfer domains between owners
		
		\item An `owner' in this context is a public key
		
		\item These public keys can be used to verify stored DNS records by their signature records
	\end{itemize}
	
	
	\begin{figure}
		\centering
		\includegraphics[width=0.3\linewidth]{namecoin_block}
		%\caption{Contents of an individual block.}
		\label{fig:blockchain}
	\end{figure}
	
\end{frame}



\begin{frame}{Example Record}
	\begin{center}
\includegraphics[width=\linewidth]{figs/record}
\end{center}

\end{frame}



\subsection{DNS Frontend}
\begin{frame}{PowerDNS}
\begin{itemize}
	\item Well established authoritative DNS server software.	
	\item Provides easy interface for custom applications.
	\item Serves the DNS requests for DHT client.
\end{itemize}

	\begin{center}
		\includegraphics[width=\linewidth]{"figs/D3NS Flowchart"}
	\end{center}


	

	
\end{frame}


\subsection{Operation}



\begin{frame}{DNS Registration}
	\begin{itemize}
		\item New domain names can be:
		\begin{itemize}
			\item Awarded as part of mining process
			\item Bought as a voucher
		\end{itemize}
		\item Domain names can be transferred between owners by creating a new record
		\item These transactions are recorded in the blockchain
	\end{itemize}	
\end{frame}

\subsection{Security}

\begin{frame}{Ramifications}

\begin{itemize}
	\item More resilient against DDOS attacks
	\begin{itemize}
		\item No top of hierarchy to attack
		\item Attacker needs to target large number of servers
	\end{itemize}
	\item Decentralization of authentication
	\begin{itemize}
		\item Authentication is baked into replication
		\item Changes to a record must be signed by the owner
	\end{itemize}
\end{itemize}	
	
	
\end{frame}


%\begin{frame}{Man in the Middle In a DHT}
%	
%	\begin{itemize}
%		\item Need to have a distributed, reliable way to authenticate 
%		\item Given:
%		\begin{itemize}
%			\item An existing network where nodes have exchanged keys securely
%			\item A new peer who wishes to join the network and share their public key
%			
%		\end{itemize}
%	\end{itemize}
%	
%	
%\end{frame}
%
%
%
%\begin{frame}{Prevention}
%	
%	\begin{itemize}
%		\item  At least 2 members of the network interrogate the new peer for its public key
%		\item  Those interrogators compare their results
%		\item  If those results match:
%		\begin{itemize}
%			
%			\item The new peer creates an authentication record
%			\item The interrogators sign that record
%			\item The new record is distributed across the network
%		\end{itemize}
%		
%		
%		\item  If the results do not match:
%		\begin{itemize}
%			
%			\item An attack is detected and reported to the new peer by all authenticating servers.
%			\item A member of the network may make a ban of the compromised peer
%			\item Otherwise the joining process can be repeated.
%		\end{itemize}
%		
%	\end{itemize}
%	
%	
%\end{frame}
%


\section{Conclusion}



\subsection{Conclusion}
\begin{frame}{Conclusions}
	\begin{itemize}
		\item Deployable prototype of a decentralized and distributed top-level DNS
		\item Stronger robustness
		\item Fully reverse compatible
		\item Offers decentralized authentication
		\item Improvements to latency
		\item Any organization can create their own secure verification server
		
	\end{itemize}
\end{frame}



\begin{frame}{Future Work}
	\begin{itemize}
		\item Optimize caching
		\item Handle backup comorbidity 
		\item Further security measures
		\item Larger scale implementation 
	\end{itemize}
\end{frame}



%
%\section*{Appendix}
%\begin{frame}{How Does This Differ From Namecoin?}
%	\begin{itemize}
%		\item Transparent to end users.
%		\item Namecoin doesn't scale efficiently, our system.
%	\end{itemize}
%\end{frame}


\end{document}
