% !TEX TS-program = pdflatex
% !TEX encoding = UTF-8 Unicode

% This file is a template using the "beamer" package to create slides for a talk or presentation
% - Giving a talk on some subject.
% - The talk is between 15min and 45min long.
% - Style is ornate.

% MODIFIED by Jonathan Kew, 2008-07-06
% The header comments and encoding in this file were modified for inclusion with TeXworks.
% The content is otherwise unchanged from the original distributed with the beamer package.

\documentclass{beamer}


% Copyright 2004 by Till Tantau <tantau@users.sourceforge.net>.
%
% In principle, this file can be redistributed and/or modified under
% the terms of the GNU Public License, version 2.
%
% However, this file is supposed to be a template to be modified
% for your own needs. For this reason, if you use this file as a
% template and not specifically distribute it as part of a another
% package/program, I grant the extra permission to freely copy and
% modify this file as you see fit and even to delete this copyright
% notice. 


\mode<presentation>
{
  \usetheme{Warsaw}
  % or ...

  \setbeamercovered{transparent}
  % or whatever (possibly just delete it)
}


\usepackage[english]{babel}
% or whatever

\usepackage[utf8]{inputenc}
% or whatever

\usepackage{times}
\usepackage[T1]{fontenc}
% Or whatever. Note that the encoding and the font should match. If T1
% does not look nice, try deleting the line with the fontenc.


\title[Project Proposal] % (optional, use only with long paper titles)
{Distributed DNS and Authentication Using DHTs}

\author % (optional, use only with lots of authors)
{Brendan Benshoof and Andrew Rosen}
% - Use the \inst{?} command only if the authors have different
%   affiliation.


\date{9/17/2013}

\subject{Talks}
% This is only inserted into the PDF information catalog. Can be left
% out. 



% If you have a file called "university-logo-filename.xxx", where xxx
% is a graphic format that can be processed by latex or pdflatex,
% resp., then you can add a logo as follows:

% \pgfdeclareimage[height=0.5cm]{university-logo}{university-logo-filename}
% \logo{\pgfuseimage{university-logo}}





% If you wish to uncover everything in a step-wise fashion, uncomment
% the following command: 

%\beamerdefaultoverlayspecification{<+->}


\begin{document}

\begin{frame}
  \titlepage
\end{frame}


% Since this a solution template for a generic talk, very little can
% be said about how it should be structured. However, the talk length
% of between 15min and 45min and the theme suggest that you stick to
% the following rules:  

% - Exactly two or three sections (other than the summary).
% - At *most* three subsections per section.
% - Talk about 30s to 2min per frame. So there should be between about
%   15 and 30 frames, all told.

\section{Introduction}

\begin{frame}{What}
\begin{itemize}
	\item We want to create a distributed Domain Name Service.
	\item Requires a authentication service that works over a distributed service.
\end{itemize}
\end{frame}


\begin{frame}{Why}
\begin{itemize}
	\item Recent events have demonstrated that centralized authorities are not as secure a previously hoped.
	\begin{itemize}
		\item There is little cryptographic protection against the subpoena.
	\end{itemize}
	\item A distributed approach for authentication is much less vulnerable the type of attacks

\end{itemize}
\end{frame}




\subsection{Solution}

\section{Components}

\section{Project}


\end{document}


