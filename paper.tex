
\documentclass{IEEEtran}

\author{ Brendan Benshoof}
\author{Andrew Rosen}
\author{
  Benshoof, Brendan\\
  \texttt{bbenshoof@cs.gsu.edu}
  \and\\
  Rosen, Andrew\\
  \texttt{rosen@cs.gsu.edu}
}
\title{P2P DNS and Signature Authority}
\begin{document}
\maketitle
\subsection{What?}

We propose a method of distributing Domain Name Services across many or
all computers on the Internet. To facilitate the authentication of the
records in this new DNS system and to protect against man in the middle
attacks, we propose a means of creating a distributed signing authority
for SSL certificates.

\subsection{Why?}

Recent events have yielded new understanding that the mutually trusted
third party used for most online key exchanges, specifically corporate
signing authorities based in the United States or countries with similar
active surveillance programs, have been compromised by the adversary and
are being used to facilitate man in the middle attacks. In order to
prevent these attacks, we need a method of authenticating domain name
records and signatures those domains use to secure communications that
are difficult to be systematically subverted by any government or
corporation.

\subsection{How?}

\begin{itemize}
\item
  Use a Distributed Hash Table (DHT) to organize a P2P network (Chord)
\item
  Use a variant on NameCoin to secure a list of servers and public keys
  and to record DNS registration and transfers.
\item
  Use the DHT to load balance and distribute responsibility for hosting
  DNS records and certification keys.
\end{itemize}
\subsection{Distributed Hash Table (Chord)}

\begin{itemize}
\item
  Means of organizing communication and responsibility in a P2P network
\item
  Each peer is responsible for a verifiable span of hash values
\item
  Facilitates one-to-one communication and one-to-many communication
\end{itemize}
\subsection{Namecoin Variant}

\begin{itemize}
\item
  Allows for a shared, immutable and secure public records
\item
  Based on the block chain verification of Bitcoin
\item
  One block can include the validation of a new server's public key
\item
  One block can include a DNS record or change
\item
  Blocks require a proof of work to authenticate, causing records to be
  produced at a semi-fixed rate
\item
  Unlike Bitcoin there is one transaction per block. Thus mining only
  happens when a new record is required
\end{itemize}
\subsection{Man in the Middle Prevention}

\begin{itemize}
\item
  Given: an existing network where nodes have exchanged keys securely
\item
  Given: a new peer who wishes to join the network and share their
  public key
  \begin{itemize}
  \item
    At least 2 members of the network interrogate the new peer for its
    public key
  \item
    Those peers that interrogated the new peer compare their results
  \item
    If those results match
    \begin{itemize}
    \item
      The new peer mines a block with an authentication record
    \item
      The peers who authenticated the new peer sign that block
    \item
      The new block is distributed across the network
    \end{itemize}
  \item
    If the results do not match
    \begin{itemize}
    \item
      An attack is detected and reported to the new peer by all
      authenticating servers
    \item
      A member of the network may mine a block with a ban of the
      compromised peer
    \item
      Otherwise the joining process may be repeated in hopes of success
    \end{itemize}
  \end{itemize}
\end{itemize}
\subsection{Distribution of DNS}

\begin{itemize}
\item
  Responsibility for serving DNS records is distributed across the
  network
\item
  Each node of the network acts as a DNS server reverse compatible with
  the DNS RFC
\item
  Any end user who wishes to use this DNS network sets any node as their
  DNS server (and ideally this node is nearby to the client)
\item
  Each node keeps a local hosts file that caches the results of recent
  and frequent results
\item
  If a node does not have the DNS record for a request locally or stored
  in the cache, it may either internally seek the value or return its
  best peer for that record, depending on the recursive bit of the DNS
  request.
\item
  Optionally, if a DNS request is for a domain the P2P DNS is not
  configured to manage, the request is forwarded to a conventional DNS
  server
\end{itemize}

\end{document}

